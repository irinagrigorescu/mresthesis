\maketitle
\makedeclaration

\begin{abstract} % 300 word limit
Magnetic Resonance Imaging (MRI) is an imaging technique which is widely used clinically as it is non-invasive and uses non-ionising radiation. However, MRI has a few potential drawbacks which include motion artifacts caused by patient movement and the high costs involved with every scan. In order to mitigate these problems, a collection of acquisition and reconstruction algorithms called parallel imaging techniques (pMRI) are clinically used in conjunction with already established MRI sequences in order to reduce scan time. However, as with any other MRI techniques, these are also prone to a handful of problems which need to be investigated in a controlled and reliable manner. The  only  way  of  accurately  assessing  the  reconstruction schemes which are used in pMRI is to simulate the parallel imaging acquisition pipeline.  As a result, the main aim of this thesis is to enhance an already existing MRI simulator called POSSUM with parallel imaging capabilities and to evaluate a popular image based reconstruction algorithm called SENSE under various combinations of parameters. For this, three objectives were met. First, multi-coil capabilities were added to POSSUM. This is an important first step as pMRI techniques rely on the sensitivity profiles of the phased-array receiver coils. Second, a parallel imaging pipeline was simulated with the enhanced version of POSSUM. Using  the  newly  enhanced  POSSUM, simulations of  increasing  acceleration  factors  were  performed for multiple types of coil geometries and sensitivity profiles. Finally, the quality of SENSE reconstructions and its robustness to noise was evaluated for various acceleration factors. The work has the potential to be extended and integrated seamlessly into the proposed simulator environment. %Moreover, it can be paired with diffusion MRI simulations to more accurately represent underlying brain microstructure.





\end{abstract}

\begin{acknowledgements}
First of all I would like to thank my supervisor Dr. Gary Hui Zhang for his constant support, invaluable guidance and all of our debugging and problem solving sessions. I would also like to thank Dr. Ivana Drobnjak for all the hours she spent explaining MR simulation concepts and for her constant moral support. Special thanks goes to my colleague and friend, Danny Raj Ramasawmy, for all those times he helped me disentangle a problem. I would also like to thank my fiance, Tudor, for being there for me every time I needed him to be. Finally, I would like to thank both my families, the one back home and the one I found here within the CDT group. 
\end{acknowledgements}

\setcounter{tocdepth}{2} 
% Setting this higher means you get contents entries for
%  more minor section headers.

\tableofcontents
\listoffigures
\listoftables

